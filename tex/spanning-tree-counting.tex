
\textbf{Kirchhoff's Theorem}: the number of spanning trees in a graph G is equal to \emph{any} cofactor of the Laplacian matrix of G, which is equal to the difference between the graph's degree matrix (a diagonal matrix with vertex degrees on the diagonals) and its adjacency matrix (a (0,1)-matrix with 1's at places corresponding to entries where the vertices are adjacent and 0's otherwise).

The number of edges with a certain weight in a minimum spanning tree is fixed given a graph. Moreover, the number of its arrangements can be obtained by finding a minimum spannig tree, compressing connected components of other edges in that tree into a point, and then applying Kirrchoff's theorem with only edges of the certain weight in the graph. Therefore, the number of minimum spanning trees in a graph can be solved by multiplying all numbers of arrangements of edges of different weight together.

