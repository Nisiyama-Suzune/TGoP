\documentclass[UTF8,a4paper]{report}

\usepackage{amsmath}
\usepackage{array}
\usepackage{courier}
\usepackage[margin=1cm, includefoot]{geometry}
\usepackage{listings}
\usepackage{mathtools}
\usepackage{multirow}
\usepackage{physics}
\usepackage{verbatim}
\usepackage{xcolor}

%\setsecnumdepth{subsubsection}
\setcounter{secnumdepth}{3}

\DeclarePairedDelimiter\ceil{\lceil}{\rceil}
\DeclarePairedDelimiter\floor{\lfloor}{\rfloor}

\newcolumntype{P}[1]{>{\centering\arraybackslash}p{#1}}
\newcolumntype{M}[1]{>{\centering\arraybackslash}m{#1}}

\lstset {
	language=C++,
	tabsize=4,
	basicstyle=\linespread{0.1}\footnotesize\ttfamily\bfseries,
	identifierstyle=\color{violet},
	keywordstyle=\color{blue},
	stringstyle=\color{brown},
	commentstyle=\color{teal}
}

\title{The Grimoire of Programming}
\date{}
\author{Suzune Nisiyama}

\begin{document}
	\tableofcontents
	\cleardoublepage
	\maketitle
	\chapter{Trial of data structure}
		\section{KD tree}
			\lstinputlisting[firstline=5,lastline=150]{../src/data-structure.cpp}
		\section{Link cut tree}
			\lstinputlisting[firstline=152,lastline=255]{../src/data-structure.cpp}
	\chapter{Trial of number theory}
		\section{Constants and basic functions}
			\lstinputlisting[firstline=9,lastline=76]{../src/number.cpp}
		\section{Discrete Fourier transform}
			\lstinputlisting[firstline=75,lastline=122]{../src/number.cpp}
		\section{Number-theoretic transform}
			\lstinputlisting[firstline=121,lastline=188]{../src/number.cpp}
		\section{Chinese remainder theorem}
			\lstinputlisting[firstline=190,lastline=223]{../src/number.cpp}
		\section{Baby step giant step algorithm}
			\lstinputlisting[firstline=339,lastline=366]{../src/number.cpp}
		\section{Miller Rabin primality test}
			\lstinputlisting[firstline=225,lastline=253]{../src/number.cpp}
		\section{Pollard's Rho algorithm}
			\lstinputlisting[firstline=255,lastline=313]{../src/number.cpp}
		\section{Adaptive Simpson's method}
			\lstinputlisting[firstline=314,lastline=337]{../src/number.cpp}
	\chapter{Trial of geometry}
		\section{Constants and basic functions}
			\lstinputlisting[firstline=10,lastline=24]{../src/geometry.cpp}
		\section{Point class}
			\lstinputlisting[firstline=26,lastline=96]{../src/geometry.cpp}
		\section{Line class}
			\lstinputlisting[firstline=98,lastline=107]{../src/geometry.cpp}
		\section{Interactions between points and lines}
			\lstinputlisting[firstline=109,lastline=181]{../src/geometry.cpp}
		\section{Centers of a triangle}
			\lstinputlisting[firstline=183,lastline=200]{../src/geometry.cpp}
		\section{Fermat point}
			\lstinputlisting[firstline=202,lastline=225]{../src/geometry.cpp}
		\section{Circle class}
			\lstinputlisting[firstline=227,lastline=243]{../src/geometry.cpp}
		\section{Interactions of circles}
			\lstinputlisting[firstline=245,lastline=297]{../src/geometry.cpp}
		\section{Convex hull}
			\lstinputlisting[firstline=299,lastline=335]{../src/geometry.cpp}
		\section{Minimum circle}
			\lstinputlisting[firstline=337,lastline=355]{../src/geometry.cpp}
		\section{Half plane intersection}
			\lstinputlisting[firstline=357,lastline=440]{../src/geometry.cpp}
		\section{Intersection of a polygon and a circle}
			\lstinputlisting[firstline=442,lastline=492]{../src/geometry.cpp}
		\section{Union of circles}	
			\lstinputlisting[firstline=494,lastline=593]{../src/geometry.cpp}
	\chapter{Trial of graph}
		\section{Constants and edge lists}
			\lstinputlisting[firstline=9,lastline=81]{../src/graph.cpp}
		\section{SPFA improved}
			\lstinputlisting[firstline=83,lastline=134]{../src/graph.cpp}
		\section{Dijkstra's shortest path algorithm}
			\lstinputlisting[firstline=136,lastline=168]{../src/graph.cpp}
		\section{Tarjan}
			\lstinputlisting[firstline=170,lastline=208]{../src/graph.cpp}
		\section{Vertex biconnected component}
			\lstinputlisting[firstline=862,lastline=902]{../src/graph.cpp}
		\section{Edge biconnected component}
			\lstinputlisting[firstline=904,lastline=942]{../src/graph.cpp}
		\section{Hopcoft-Carp}
			\lstinputlisting[firstline=210,lastline=269]{../src/graph.cpp}
		\section{Kuhn-Munkres}
			\lstinputlisting[firstline=271,lastline=331]{../src/graph.cpp}
		\section{Stochastic weighted maximum matching}
			\lstinputlisting[firstline=333,lastline=401]{../src/graph.cpp}
		\section{Weighted blossom (vfleaking ver.)}	
			\lstinputlisting[firstline=403,lastline=617]{../src/graph.cpp}
		\section{Maximum flow}
			\lstinputlisting[firstline=619,lastline=732]{../src/graph.cpp}
		\section{Minimum cost flow}
			\lstinputlisting[firstline=734,lastline=860]{../src/graph.cpp}
		\section{Dominator tree}
			\lstinputlisting[firstline=944,lastline=1030]{../src/graph.cpp}
	\chapter{Trial of string}
		\section{KMP}
			\lstinputlisting[firstline=9,lastline=42]{../src/string.cpp}
		\section{Suffix array}
			\lstinputlisting[firstline=163,lastline=214]{../src/string.cpp}
		\section{Suffix automaton}
			\lstinputlisting[firstline=44,lastline=102]{../src/string.cpp}
		\section{Palindromic tree}
			\lstinputlisting[firstline=105,lastline=161]{../src/string.cpp}
	\chapter{Reference}
		\section{Vimrc}
			\begin{lstlisting}[language=tex,identifierstyle=\color{black},commentstyle=\color{black}]
set ruler
set number
set tabstop=4
set softtabstop=4
set shiftwidth=4
set smartindent
set showmatch
set hlsearch
set incsearch
set autoread
set backspace=2
set mouse=a

syntax on

nmap <F3> :vsplit %<.in <CR>
nmap <F4> :!gedit % <CR>

autocmd FileType cpp map <F5> :!time ./%< <CR>
autocmd FileType cpp map <F7> :!gdb %< <CR>
autocmd FileType cpp map <F8> :!time ./%< < %<.in <CR>
autocmd FileType cpp map <F9> :!g++ % -o %< -std=c++11 && size %< <CR>

autocmd FileType java map <F5> :!time java %< <CR>
autocmd FileType java map <F8> :!time java %< < %<.in <CR>
autocmd FileType java map <F9> :!javac % <CR>
\end{lstlisting}


		\section{Java reference}
			\lstinputlisting[language=Java]{../src/Main.java}
		\section{Operator precedence}
			\begin{center}
	\begin{tabular}{|c|c|c|c|}
		\hline
		Precedence			&	Operator					&	Description												&	Associativity					\\
		\hline
		1					&	\ttfamily ::				&	Scope resolution										&	\multirow{6}{*}{Left-to-right}	\\
		\cline{1-3}
		\multirow{5}{*}{2}	&	\ttfamily a++ a--			&	Suffix/postfix increment and decrement					&									\\
							&	\ttfamily type() type\{\}	&	Functional cast											&									\\
							&	\ttfamily a()				&	Function call											&									\\
							&	\ttfamily a[]				&	Subscript												&									\\
							&	\ttfamily . ->				&	Member access											&									\\
		\hline	
		\multirow{9}{*}{3}	&	\ttfamily ++a --a			&	Prefix increment and decrement							&	\multirow{9}{*}{Right-to-left}	\\
							&	\ttfamily +a -a				&	Unary plus and minus									&									\\
							&	\ttfamily ! \~				&	Logical NOT and bitwise NOT								&									\\
							&	\ttfamily (type)			&	C-style cast											&									\\
							&	\ttfamily *a				&	Indirection (dereference)								&									\\
							&	\ttfamily \&a				&	Address-of												&									\\
							&	\ttfamily sizeof			&	Size-of													&									\\
							&	\ttfamily new new[]			&	Dynamic memory allocation								&									\\
							&	\ttfamily delete delete[]	&	Dynamic memory deallocation								&									\\
		\hline
		4					&	\ttfamily .* ->*			&	Pointer-to-member										&	\multirow{12}{*}{Left-to-right}	\\
		\cline{1-3}
		5					&	\ttfamily a*b a/b a\%b		&	Multiplication, division, and remainder					&									\\
		\cline{1-3}
		6					&	\ttfamily a+b a-b			&	Addition and subtraction								&									\\
		\cline{1-3}
		7					&	\ttfamily << >>				&	Bitwise left shift and right shift						&									\\
		\cline{1-3}
		\multirow{2}{*}{8}	&	\ttfamily < <=				&	For relational operators $<$ and $\leq$ respectively	&									\\
							&	\ttfamily > >=				&	For relational operators $>$ and $\geq$ respectively	&									\\
		\cline{1-3}
		9					&	\ttfamily == !=				&	For relational operators $=$ and $\neq$ respectively	&									\\
		\cline{1-3}
		10					&	\ttfamily a\&b				&	Bitwise AND												&									\\
		\cline{1-3}
		11					&	\ttfamily \^{}				&	Bitwise XOR (exclusive or)								&									\\
		\cline{1-3}
		12					&	\ttfamily |					&	Bitwise OR (inclusive or)								&									\\
		\cline{1-3}
		13					&	\ttfamily \&\&				&	Logical AND												&									\\
		\cline{1-3}
		14					&	\ttfamily ||				&	Logical OR												&									\\
		\hline
		\multirow{6}{*}{15}	&	\ttfamily a?b:c				&	Ternary conditional										&	\multirow{6}{*}{Right-to-left}	\\
							&	\ttfamily throw				&	throw operator											&									\\
							&	\ttfamily =					&	Direct assignment										&									\\
							&	\ttfamily += -=	*= /= \%=	&	Compound assignment by arithmetic operation				&									\\
							&	\ttfamily <<= >>=			&	Compound assignment by bitwise shift					&									\\
							&	\ttfamily \&= \^{}= |=		&	Compound assignment by bitwise AND, XOR, and OR			&									\\
		\hline
		16					&	\ttfamily ,					&	Comma													&	Left-to-right					\\
		\hline
	\end{tabular}
\end{center}


		\section{Hacks}
			\subsection{Ultra fast functions}
				\lstinputlisting[firstline=30,lastline=72]{../src/hack.cpp}
			\subsection{Formating long long in scanf \& printf}
				\lstinputlisting[firstline=7,lastline=11]{../src/hack.cpp}
			\subsection{Optimizing}
				\lstinputlisting[firstline=15,lastline=16]{../src/hack.cpp}
			\subsection{Larger stack}
				\subsubsection{C++}
					\lstinputlisting[firstline=22,lastline=22]{../src/hack.cpp}
				\subsubsection{G++}
					\lstinputlisting[firstline=26,lastline=28]{../src/hack.cpp}
		\section{Math reference}
			\subsection{Catalan number}
				
For

\begin{align*}
	f(0)&=1\\
	f(1)&=1\\
	f(n)&=f(n-1)f(0)+f(n-2)f(1)+...+f(1)f(n-2)+f(0)f(n-1)\\
\end{align*}

We have $f(n)=\frac{(2n)!}{n!(n+1)!}$.


			\subsection{Dynamic programming optimization}
				\subsubsection{Convex hull optimization}
	Generally, in dynamic programming with recurrence
	$$f(i) = \min_{k<i}\{a[i]b(j)+c(j)\}$$
	all decisions $k$ can be treated as a set of segments on a convex hull.
	By applying Graham's scanning, it is possible to maintain such hull in a monotone queue or a \verb!std::tuple <slope, intercept, x_min>!.
	Hence, $k(i)$ can be obtained by performing a binary search in the hull.
\subsubsection{Divide \& conquer optimization}
	For recurrence 
	$$f(i) = \min_{k<i}\{b(k)+c[k][i]\}$$
	$k(i) \leq k(i+1)$ holds true if $c[a][c]+c[b][d]<c[a][d]+c[b][c]$.
	Thus, $k(i)$ can be maintained in a monotone queue.
\subsubsection{Knuth optimization}
	For recurrence
	$$f(i,j) = \min_{i<k<j}\{f(i,k)+f(k,j)\}+c[i][j]$$
	$k(i,j-1) \leq k(i,j) \leq k(i+1,j)$ holds true if $c[a][c]+c[b][d]<c[a][d]+c[b][c]$.


			\subsection{Integration table}
				\subsubsection{$ax^2+bx+c\ (a>0)$}
	\begin{enumerate}
		\item$\int\frac{\dd{x}}{ax^2+bx+c}=\begin{cases}
			\frac{2}{\sqrt{4ac-b^2}}\arctan\frac{2ax+b}{\sqrt{4ac-b^2}}+C							&b^2<4ac\\
			\frac{1}{\sqrt{b^2-4ac}}\ln\abs{\frac{2ax+b-\sqrt{b^2-4ac}}{2ax+b+\sqrt{b^2-4ac}}}+C	&b^2>4ac\\
		\end{cases}$
		\item$\int\frac{x}{ax^2+bx+c}\dd{x}=\frac{1}{2a}\ln\abs{ax^2+bx+c}-\frac{b}{2a}\int\frac{\dd{x}}{ax^2+bx+c}$
	\end{enumerate}
\subsubsection{$\sqrt{\pm ax^2+bx+c}\ (a>0)$}
	\begin{enumerate}
		\item$\int\frac{\dd{x}}{ax^2+bx+c}=\frac{1}{\sqrt{a}}\ln\abs{2ax+b+2\sqrt{a}\sqrt{ax^2+bx+c}}+C$
		\item$\int\sqrt{ax^2+bx+c}\dd{x}=\frac{2ax+b}{4a}\sqrt{ax^2+bx+c}+\frac{4ac-b^2}{8\sqrt{a^3}}\ln\abs{2ax+b+2\sqrt{a}\sqrt{ax^2+bx+c}}+C$
		\item$\int\frac{x}{\sqrt{ax^2+bx+c}}\dd{x}=\frac{1}{a}\sqrt{ax^2+bx+c}-\frac{b}{2\sqrt{a^3}}\ln\abs{2ax+b+2\sqrt{a}\sqrt{ax^2+bx+c}}+C$
		\item$\int\frac{\dd{x}}{\sqrt{-ax^2+bx+c}}=-\frac{1}{\sqrt{a}}\arcsin\frac{2ax-b}{\sqrt{b^2+4ac}}+C$
		\item$\int\sqrt{-ax^2+bx+c}\dd{x}=\frac{2ax-b}{4a}\sqrt{-ax^2+bx+c}+\frac{b^2+4ac}{8\sqrt{a^3}}\arcsin\frac{2ax-b}{\sqrt{b^2+4ac}}+C$
		\item$\int\frac{x}{\sqrt{-ax^2+bx+c}}\dd{x}=-\frac{1}{a}\sqrt{-ax^2+bx+c}+\frac{b}{2\sqrt{a^3}}\arcsin\frac{2ax-b}{\sqrt{b^2+4ac}}+C$
	\end{enumerate}
\subsubsection{Trigonometric}
	\begin{enumerate}
		\item$\int\tan x\dd{x}=-\ln\abs{\cos x}+C$
		\item$\int\cot x\dd{x}=\ln\abs{\sin x}+C$
		\item$\int\sec x\dd{x}=\ln\abs{\tan(\frac{\pi}{4}+\frac{x}{2})}+C=\ln\abs{\sec x+\tan x}+C$
		\item$\int\csc x\dd{x}=\ln\abs{\tan\frac{x}{2}}+C=\ln\abs{\csc x-\cot x}+C$
		\item$\int\sec^2x\dd{x}=\tan x+C$
		\item$\int\csc^2x\dd{x}=-\cot x+C$
		\item$\int\sec x \tan x\dd{x}=\sec x+C$
		\item$\int\csc x \cot x\dd{x}=-\csc x+C$
		\item$\int\sin^2x\dd{x}=\frac{x}{2}-\frac{1}{4}\sin 2x+C$
		\item$\int\cos^2x\dd{x}=\frac{x}{2}+\frac{1}{4}\sin 2x+C$
		\item$\int\sin^nx\dd{x}=-\frac{1}{n}\sin^{n-1}x\cos x+\frac{n-1}{n}\int\sin^{n-2}x\dd{x}$
		\item$\int\cos^nx\dd{x}=\frac{1}{n}\cos^{n-1}x\sin x+\frac{n-1}{n}\int cos^{n-2}x\dd{x}$
		\item$\int\frac{\dd{x}}{\sin^nx}=-\frac{1}{n-1}\frac{\cos x}{\sin^{n-1}x}+\frac{n-2}{n-1}\int\frac{\dd{x}}{\sin^{n-2}x}$
		\item$\int\frac{\dd{x}}{\cos^nx}=\frac{1}{n-1}\frac{\sin x}{\cos^{n-1}x}+\frac{n-2}{n-1}\int\frac{\dd{x}}{\cos^{n-2}x}$
		\item$\int\cos^m x\sin^n x\dd{x}=\frac{1}{m+n}\cos^{m-1}x\sin^{n+1}x+\frac{m-1}{m+n}\int\cos^{m-2}x\sin^n x\dd{x}=-\frac{1}{m+n}\cos^{m+1}x\sin^{n-1}x+\\\frac{n-1}{m+1}\int\cos^m x\sin^{n-2}x\dd{x}$
	\end{enumerate}
\subsubsection{Inverse trigonometric$\ (a>0)$}
	\begin{enumerate}
		\item$\int\arcsin\frac{x}{a}\dd{x}=x\arcsin\frac{x}{a}+\sqrt{a^2-x^2}+C$
		\item$\int\arccos\frac{x}{a}\dd{x}=x\arccos\frac{x}{a}-\sqrt{a^2-x^2}+C$
		\item$\int\arctan\frac{x}{a}\dd{x}=x\arctan\frac{x}{a}-\frac{a}{2}\ln(a^2+x^2)+C$
	\end{enumerate}
\subsubsection{Exponential}
	\begin{enumerate}
		\item$\int a^x\dd{x}=\frac{1}{\ln a}a^x+C$
		\item$\int e^{ax}\dd{x}=\frac{1}{a}a^{ax}+C$
	\end{enumerate}
\subsubsection{Logistic}
	\begin{enumerate}
		\item$\int\ln x\dd{x}=x\ln x-x+C$
		\item$\frac{\dd{x}}{x\ln x}=\ln\abs{\ln x}+C$
	\end{enumerate}

			\subsection{Prefix sum of multiplicative functions}
				Define the Dirichlet convolution $f*g(n)$ as:

$$f*g(n)=\sum^n_{d=1}[d|n]f(n)g(\frac{n}{d})$$

Assume we are going to calculate some function $S(n)=\sum^n_{i=1}f(i)$,
where $f(n)$ is a multiplicative function.
Say we find some $g(n)$ that is simple to calculate,
and $\sum^n_{i=1}f*g(i)$ can be figured out in $O(1)$ complexity.
Then we have

\begin{equation*}
\begin{split}
\sum^n_{i=1}f*g(i)	&=\sum^n_{i=1}\sum_d[d|i]g(\frac{i}{d})f(d)\\
					&=\sum^n_{\frac{i}{d}=1}\sum^{\floor*{\frac{n}{\frac{i}{d}}}}_{d=1}g(\frac{i}{d})f(d)\\
					&=\sum^n_{i=1}\sum^{\floor*{\frac{n}{i}}}_{d=1}g(i)f(d)\\
					&=g(1)S(n)+\sum^n_{i=2}g(i)S(\floor*{\frac{n}{i}})\\
S(n)				&=\frac{\sum^n_{i=1}f*g(i)-\sum^n_{i=2}g(i)S(\floor*{\frac{n}{i}})}{g(1)}\\
\end{split}
\end{equation*}


It can be proven that $\floor*{\frac{n}{i}}$ has at most $O(\sqrt{n})$ possible values.
Therefore, the calculation of $S(n)$ can be reduced to $O(\sqrt{n})$ calculations of $S(\floor*{\frac{n}{i}})$.
By applying the master theorem, it can be shown that the complexity of such method is $O(n^{\frac{3}{4}})$.

Moreover, since $f(n)$ is multiplicative, we can process the first $n^{\frac{2}{3}}$ elements via linear sieve,
and for the rest of the elements, we apply the method shown above. The complexity can thus be enhaced to $O(n^{\frac{2}{3}})$.

For the prefix sum of Euler's function $S(n)=\sum^n_{i=1}\varphi(i)$, notice that $\sum_{d|n}\varphi(d)=n$.
Hence $\varphi*I(n)=id(n)$.($I(n)=1,id(n)=n$)
Now let $g(n)=I(n)$, and we have $S(n)=\sum^n_{i=1}i-\sum^n_{i=2}S(\floor*{\frac{n}{i}})$.

For the prefix sum of Mobius function $S(n)=\sum^n_{i=1}\mu(i)$, notice that $\mu*I(n)=[n=1]$.
Hence $S(n)=1-\sum^n_{i=2}S(\floor*{\frac{n}{i}})$.

Example code :

\begin{lstlisting}
/*	Prefix sum of multiplicative functions :
		CUBEN : N ^ (1 / 3).
		p_f : the prefix sum of f (x) (1 <= x <= th).
		p_g : the prefix sum of g (x) (0 <= x <= N).
		p_c : the prefix sum of f (x) * g (x) (0 <= x <= N).
		th : the thereshold, generally should be x ^ (2 / 3).
		REMEMBER THAT x IN p_g (x) AND p_c (x) MAY BE LARGER THAN MOD!!
*/

template <int CUBEN = 11000>
struct prefix_mul {

	typedef long long (*func) (long long);

	func p_f, p_g, p_c;
	long long n, mod, th, inv;
	std::pair <bool, long long> mem[CUBEN];

	prefix_mul (func p_f, func p_g, func p_c) : p_f (p_f), p_g (p_g), p_c (p_c) {}

	void euclid (long long a, long long b, long long &x, long long &y) {
		if (b == 0) x = 1, y = 0;
		else euclid (b, a % b, y, x), y -= a / b * x;
	}

	long long inverse (long long x, long long m) {
		long long a, b;
		euclid (x, m, a, b);
		return (a % m + m) % m;
	}

	long long calc (long long x) {
		if (x <= th) return p_f (x);
		if (mem[n / x].first) return mem[n / x].second;
		mem[n / x].first = true;
		long long ans = 0;
		for (long long i = 2, la; i <= x; i = la + 1) {
			la = x / (x / i);
			ans = (ans + (p_g (la) - p_g (i - 1) + mod) * calc (x / i)) % mod;
		}
		ans = p_c (x) - ans; if (ans < 0) ans += mod; ans = ans * inv % mod;
		return mem[n / x].second = ans;
	}

	long long solve (long long n, long long th, long long mod) {
		if (n <= 0) return 0;
		prefix_mul::n = n; prefix_mul::mod = mod; prefix_mul::th = th;
		inv = inverse (p_g (1), mod);
		std::fill (mem, mem + CUBEN, std::make_pair (false, 0ll));
		return calc (n); 
	}

};
\end{lstlisting}


			\subsection{Prufer sequence}
				
In combinatorial mathematics, the Prufer sequence of a labeled tree is a unique sequence associated with the tree. The sequence for a tree on $n$ vertices has length $n-2$.

One can generate a labeled tree's Prufer sequence by iteratively removing vertices from the tree until only two vertices remain. Specifically, consider a labeled tree $T$ with vertices ${1, 2, ..., n}$. At step $i$, remove the leaf with the smallest label and set the $i$th element of the Prufer sequence to be the label of this leaf's neighbour.

One can generate a labeled tree from a sequence in three steps. The tree will have $n+2$ nodes, numbered from $1$ to $n+2$. For each node set its degree to the number of times it appears in the sequence plus $1$. Next, for each number in the sequence $a[i]$, find the first (lowest-numbered) node, $j$, with degree equal to $1$, add the edge $(j, a[i])$ to the tree, and decrement the degrees of $j$ and $a[i]$. At the end of this loop two nodes with degree $1$ will remain (call them $u$, $v$). Lastly, add the edge $(u,v)$ to the tree.

The Prufer sequence of a labeled tree on $n$ vertices is a unique sequence of length $n-2$ on the labels $1$ to $n$ - this much is clear. Somewhat less obvious is the fact that for a given sequence $S$ of length $n-2$ on the labels $1$ to $n$, there is a unique labeled tree whose Prufer sequence is $S$.


			\subsection{Spanning tree counting}
				
\textbf{Kirchhoff's Theorem}: the number of spanning trees in a graph G is equal to \emph{any} cofactor of the Laplacian matrix of G, which is equal to the difference between the graph's degree matrix (a diagonal matrix with vertex degrees on the diagonals) and its adjacency matrix (a (0,1)-matrix with 1's at places corresponding to entries where the vertices are adjacent and 0's otherwise).

The number of edges with a certain weight in a minimum spanning tree is fixed given a graph. Moreover, the number of its arrangements can be obtained by finding a minimum spannig tree, compressing connected components of other edges in that tree into a point, and then applying Kirrchoff's theorem with only edges of the certain weight in the graph. Therefore, the number of minimum spanning trees in a graph can be solved by multiplying all numbers of arrangements of edges of different weight together.


		\section{Regular expression}
			\lstinputlisting[firstline=4,lastline=15]{../src/regex.cpp}	
			\subsection{Special pattern characters}
\begin{center}
	\begin{tabular}{|M{2cm}|M{3cm}|M{12cm}|}
		\hline
		Characters			&	Description				&	Matches\\
		\hline
		\verb!.!			&	Not newline				&	Any character except line terminators (LF, CR, LS, PS).\\
		\hline
		\verb!\t!			&	Tab (HT)				&	A horizontal tab character (same as \verb!\u0009!).\\
		\hline
		\verb!\n!			&	Newline (LF)			&	A newline (line feed) character (same as \verb!\u000A!).\\
		\hline
		\verb!\v!			&	Vertical tab (VT)		&	A vertical tab character (same as \verb!\u000B!).\\
		\hline
		\verb!\f!			&	Form feed (FF)			&	A form feed character (same as \verb!\u000C!).\\
		\hline
		\verb!\r!			&	Carriage return (CR)	&	A carriage return character (same as \verb!\u000D!).\\
		\hline
		\verb!\cletter!		&	Control code			&	A control code character whose code unit value is the same as the remainder of dividing the code unit value of letter by 32. For example: \verb!\ca! is the same as \verb!\u0001!, \verb!\cb! the same as \verb!\u0002!, and so on...\\
		\hline
		\verb!\xhh!			&	ASCII character			&	A character whose code unit value has an hex value equivalent to the two hex digits hh. For example: \verb!\x4c! is the same as \verb!L!, or \verb!\x23! the same as \verb!#!.\\
		\hline
		\verb!\uhhhh!		&	Unicode character		&	A character whose code unit value has an hex value equivalent to the four hex digits hhhh.\\
		\hline
		\verb!\0!			&	Null					&	A null character (same as \verb!\u0000!).\\
		\hline
		\verb!\int!			&	Backreference			&	The result of the submatch whose opening parenthesis is the int-th (int shall begin by a digit other than 0). See groups below for more info.\\
		\hline
		\verb!\d!			&	Digit					&	A decimal digit character (same as \verb![[:digit:]]!).\\
		\hline
		\verb!\D!			&	Not digit				&	Any character that is not a decimal digit character (same as \verb![^[:digit:]]!).\\
		\hline
		\verb!\s!			&	Whitespace				&	A whitespace character (same as \verb![[:space:]]!).\\
		\hline
		\verb!\S!			&	Not whitespace			&	Any character that is not a whitespace character (same as \verb![^[:space:]]!).\\
		\hline
		\verb!\w!			&	Word					&	An alphanumeric or underscore character \verb!(same as [_[:alnum:]])!.\\
		\hline
		\verb!\W!			&	Not word				&	Any character that is not an alphanumeric or underscore character (same as \verb![^_[:alnum:]])!.\\
		\hline
		\verb!\character!	&	Character				&	The character character as it is, without interpreting its special meaning within a regex expression. Any character can be escaped except those which form any of the special character sequences above. Needed for: \verb!^ $ \ . * + ? ( ) [ ] { } |!.\\
		\hline\relax
		\verb![class]!		&	Character class			&	The target character is part of the class (see character classes below).\\
		\hline\relax
		\verb![^class]!		&	Negated character class	&	The target character is not part of the class (see character classes below).\\
		\hline
	\end{tabular}
\end{center}
\subsection{Quantifiers}
\begin{center}
	\begin{tabular}{|M{2cm}|M{3cm}|M{12cm}|}
		\hline
		Characters			&	Times								&	Effects\\
		\hline
		\verb!*!			&	0 or more							&	The preceding atom is matched 0 or more times.\\
		\hline
		\verb!+!			&	1 or more							&	The preceding atom is matched 1 or more times.\\
		\hline
		\verb!?!			&	0 or 1								&	The preceding atom is optional (matched either 0 times or once).\\
		\hline
		\verb!{int}!		&	\verb!int!							&	The preceding atom is matched exactly \verb!int! times.\\
		\hline
		\verb!{int,}!		&	\verb!int! or more					&	The preceding atom is matched \verb!int! or more times.\\
		\hline
		\verb!{min,max}!	&	Between \verb!min! and \verb!max!	&	The preceding atom is matched at least \verb!min! times, but not more than \verb!max!.\\
		\hline
	\end{tabular}
\end{center}

By default, all these quantifiers are greedy (i.e., they take as many characters that meet the condition as possible). This behavior can be overridden to ungreedy (i.e., take as few characters that meet the condition as possible) by adding a question mark (\verb!?!) after the quantifier.

\subsection{Groups}
\begin{center}
	\begin{tabular}{|M{4cm}|M{3cm}|M{10cm}|}
		\hline
		Characters				&	Description		&	Effects\\
		\hline
		\verb!(subpattern)!		&	Group			&	Creates a backreference.\\
		\hline
		\verb!(?:subpattern)!	&	Passive group	&	Does not create a backreference.\\
		\hline
	\end{tabular}
\end{center}
\subsection{Assertions}
\begin{center}
	\begin{tabular}{|M{4cm}|M{3cm}|M{10cm}|}
		\hline
		Characters				&	Description			&	Condition for match\\
		\hline
		\verb!^!				&	Beginning of line	&	Either it is the beginning of the target sequence, or follows a line terminator.\\
		\hline
		\verb!$!				&	End of line			&	Either it is the end of the target sequence, or precedes a line terminator.\\
		\hline
		\verb!\b!				&	Word boundary		&	The previous character is a word character and the next is a non-word character (or vice-versa). Note: The beginning and the end of the target sequence are considered here as non-word characters.\\
		\hline
		\verb!\B!				&	Not a word boundary	&	The previous and next characters are both word characters or both are non-word characters. Note: The beginning and the end of the target sequence are considered here as non-word characters.\\
		\hline
		\verb!(?=subpattern)!	&	Positive lookahead	&	The characters following the assertion must match subpattern, but no characters are consumed.\\
		\hline
		\verb|(?!subpattern)|	&	Negative lookahead	&	The characters following the assertion must not match subpattern, but no characters are consumed.\\
		\hline
	\end{tabular}
\end{center}
\subsection{Alternative}
A regular expression can contain multiple alternative patterns simply by separating them with the separator operator (\verb!|!): The regular expression will match if any of the alternatives match, and as soon as one does.
\subsection{Character classes}
\begin{center}
	\begin{tabular}{|M{2cm}|M{10cm}|M{5cm}|}
		\hline
		Class				&	Description								&	Equivalent (with regex\_traits, default locale)\\
		\hline
		\verb![:alnum:]!	&	Alpha-numerical character				&	\verb!isalnum!\\
		\hline
		\verb![:alpha:]!	&	Alphabetic character					&	\verb!isalpha!\\
		\hline
		\verb![:blank:]!	&	Blank character							&	\verb!isblank!\\
		\hline
		\verb![:cntrl:]!	&	Control character						&	\verb!iscntrl!\\
		\hline
		\verb![:digit:]!	&	Decimal digit character					&	\verb!isdigit!\\
		\hline
		\verb![:graph:]!	&	Character with graphical representation	&	\verb!isgraph!\\
		\hline
		\verb![:lower:]!	&	Lowercase letter						&	\verb!islower!\\
		\hline
		\verb![:print:]!	&	Printable character						&	\verb!isprint!\\
		\hline
		\verb![:punct:]!	&	Punctuation mark character				&	\verb!ispunct!\\
		\hline
		\verb![:space:]!	&	Whitespace character					&	\verb!isspace!\\
		\hline
		\verb![:upper:]!	&	Uppercase letter						&	\verb!isupper!\\
		\hline
		\verb![:xdigit:]!	&	Hexadecimal digit character				&	\verb!isxdigit!\\
		\hline
		\verb![:d:]!		&	Decimal digit character					&	\verb!isdigit!\\
		\hline
		\verb![:w:]!		&	Word character							&	\verb!isalnum!\\
		\hline
		\verb![:s:]!		&	Whitespace character					&	\verb!isspace!\\
		\hline
	\end{tabular}
\end{center}
Please note that the brackets in the class names are additional to those opening and closing the class definition. For example:

\verb![[:alpha:]]! is a character class that matches any alphabetic character.

\verb![abc[:digit:]]! is a character class that matches a, b, c, or a digit.

\verb![^[:space:]]! is a character class that matches any character except a whitespace.

\end{document}

